\documentclass[12pt]{article}
\usepackage[utf8]{inputenc}
% \usepackage{fontspec}
% \setmainfont[Ligatures=TeX]{Arial}
\newcommand{\comment}[1]{}

%\usepackage[scaled]{helvet}
%\renewcommand\familydefault{\sfdefault} 
%\usepackage[T1]{fontenc}
\usepackage{makecell}
\usepackage{hyperref}
\usepackage{lingmacros}
\usepackage{anyfontsize}
\usepackage{tabularx}
\newcolumntype{L}{>{\raggedright\arraybackslash}X}
\usepackage{adjustbox}
\usepackage{float}
\usepackage{tree-dvips}
\usepackage[a4paper,top=2cm,bottom=2cm,left=3cm,right=3cm,marginparwidth=1.75cm]{geometry}

\makeatletter
\renewcommand{\@seccntformat}[1]{}
\makeatother

\newcommand{\sign}[1]{%      
  \begin{tabular}[t]{@{}l@{}}
  \makebox[2.5in]{\dotfill}\\
  \strut#1\strut
  \end{tabular}%
}
\newcommand{\Date}{%
  \begin{tabular}[t]{@{}p{2.5in}@{}}
  \\[-2ex]
  \strut Date: \dotfill\strut
  \end{tabular}%
}

\begin{document}


\noindent
\begin{minipage}[H]{0.17\linewidth}
\centering
\includegraphics[width=1.0\linewidth]{logo.png}
\end{minipage}%
\hfill
  \begin{minipage}[H]{0.8\linewidth}
{\fontsize{28}{30}\selectfont \textbf{University of Peradeniya\\Sri Lanka}}

\Large Peradeniya, Sri Lanka 20400\\
{\fontsize{12}{30}\selectfont Telephone:+94812393301\hfill Fax:+94812388158}


  \end{minipage}

\vspace{10pt}
\noindent\rule{\textwidth}{1pt}
\vspace{-15pt}
\begin{center}
{\fontsize{21}{30}\selectfont \textbf{ACADEMIC TRANSCRIPT (INTERIM)}}
\end{center}

\vspace{-12.5pt}

\noindent\rule{\textwidth}{1pt}

\begin{table}[H]
\begin{tabularx}{\textwidth}{Xl}
\textbf{Registration Number} & E/18/147 \\
\textbf{Name in Full} & Saadia Jameel \\
\textbf{Date of Birth} & 17 June 1999 \\
% \end{tabularx}
% \end{table}
\\
\\
% \vspace{-10pt}

% \begin{table}[H]
% \begin{tabularx}{\textwidth}{Xl}
\textbf{Field of Specialization} & Computer Engineering \\
\textbf{Degree} & Bachelor of the Science of Engineering \\
\textbf{Medium of Instruction} & English \\
\textbf{Current GPA} & 4.00 \\
\end{tabularx}
\end{table}

\vspace{-15pt}

\noindent\rule{\textwidth}{1pt}

\vspace{-20pt}

\section*{General Programme in Engineering Courses}

A student who advances to follow the Specialization Programme in Engineering has earned a minimum of 33 credits from the General Programme in Engineering (This is without considering GP 102: English II).

\begin{table}[h]
\begin{tabularx}{\textwidth}{
    |>{\hsize=0.6\hsize}X| 
    >{\hsize=0.5\hsize}X|
    >{\hsize=2.0\hsize}X|
    >{\hsize=0.4\hsize}X|
    >{\hsize=0.5\hsize}X|
   }
\hline 
\textbf{Semester Ending Date} & \textbf{Course ID} & \textbf{Course Unit Name} & \textbf{Grade} & \textbf{Credits} \\ 
\hline
27-Jun-19 & GP101 & English I & A+ & 3 \\ 
\hline
27-Jun-19 & GP109 & Materials Science & A+ & 3 \\ 
\hline
27-Jun-19 & GP110 & Engineering Mechanics & A+ & 3 \\ 
\hline
27-Jun-19 & GP112 & Engineering Measurements & A+ & 3 \\ 
\hline
27-Jun-19 & GP114 & Engineering Drawing & A & 3 \\ 
\hline
27-Jun-19 & GP115 & Calculus I & A+ & 3 \\ 
\hline
06-Dec-19 & GP106 & Computing & A+ & 3 \\ 
\hline
06-Dec-19 & GP111 & Elementary Thermodynamics & A+ & 3 \\
\hline
06-Dec-19 & GP113 & Fundamentals of Manufacture & A+ & 3 \\ 
\hline
06-Dec-19 & GP116 & Linear Algebra & A+ & 3 \\ 
\hline
06-Dec-19 & GP118 & \makecell[l]{Basic Electrical \& \\Electronic Engineering} & A+ & 3 \\ 
\hline
\end{tabularx}
\end{table}

\vspace{-10pt}

\section*{Core and Technical Elective (TE) Courses}

\begin{tabularx}{\textwidth}{|X|l|}
\hline 
\textbf{Credits Offered} & 82 \\ \hline 
\textbf{Credits Earned from Core and Technical Elective courses to claim the Degree} & 82 \\ \hline 
\textbf{Credit Deficit from Core and Technical Elective courses} & 0 \\
\hline 
\textbf{GPA} & 0.00 \\
\hline 
\end{tabularx}

\noindent The following Core and Technical Elective courses contribute towards the calculation of GPA. If a course is repeated, the best attempt is used for all of the above calculations.

\begin{table}[H]
\begin{tabularx}{\textwidth}{
    |>{\hsize=1.0\hsize}X| 
    >{\hsize=0.7\hsize}X|
    >{\hsize=2.4\hsize}X|
    >{\hsize=0.6\hsize}X|
    >{\hsize=0.6\hsize}X|
    >{\hsize=0.7\hsize}X|
  }
\hline 
\textbf{Semester Ending Date} & \textbf{Course ID} & \textbf{Course Unit Name} & \textbf{Grade} & \textbf{Grade Point} & \textbf{Credits} \\ 
\hline
06-Jun-17 & CO221 & Digital Design & A+ & 0.0 & 3 \\ 
\hline
06-Jun-17 & CO222 & Programming Methodology & A+ & 0.0 & 3 \\ 
\hline
06-Jun-17 & CO223 & Computer Communication Networks I & A+ & 0.0 & 3 \\ 
\hline
06-Jun-17 & EE282 & Network Analysis for Computer Engineering & A+ & 0.0 & 3 \\
\hline
06-Jun-17 & EM211 & Ordinary Differential Equations  & A+ & 0.0 & 2 \\ 
\hline
06-Jun-17 & EM213 & Probability and Statistics & A+ & 0.0 & 2 \\ 
\hline
06-Jun-17 & EM214 & Discrete Mathematics & A+ & 0.0 & 3 \\ 
\hline
23-Oct-17 & CO224 & Computer Architecture & A+ & 0.0 & 3 \\ 
\hline
23-Oct-17 & CO225 & Software Construction & A+ & 0.0 & 3 \\ 
\hline
23-Oct-17 & CO226 & Database Systems & A+ & 0.0 & 3 \\ 
\hline
23-Oct-17 & EE285 & Electronics I & A+ & 0.0 & 3 \\ 
\hline
23-Oct-17 & EM212 & Calculus II & A+ & 0.0 & 2 \\ 
\hline
23-Oct-17 & EM215 & Numerical Methods & A+ & 0.0 & 3 \\ 
% \hline
% 23-Oct-17 & EM514 & Partial Differential Equations & Z+- & 0.0 & 3 \\ 
\hline
29-Dec-17 & CO227 & Computer Engineering Project & A+ & 0.0 & 2 \\ 
\hline

17-Sep-18 & CO321 & Embedded Systems & A+ & 0.0 & 3 \\ 
\hline
17-Sep-18 & CO322 & Data Structures \& Algorithms & A+ & 0.0 & 3 \\ 
\hline
17-Sep-18 & CO323 & Computer Communication Networks II & A+ & 0.0 & 3 \\ 
\hline
17-Sep-18 & CO324 & Network \& Web Application Design & A+ & 0.0 & 3 \\ 
\hline
17-Sep-18 & CO325 & Computer \& Network Security & A+ & 0.0 & 3 \\ 
\hline
17-Sep-18 & EE386 & Electronic Devices \& Circuits II & A+ & 0.0 & 3 \\ 
\hline
17-Sep-18 & EM503 & Graph Theory \& Circuits II & A+ & 0.0 & 3 \\ 
\hline



\end{tabularx}
\end{table}

\section*{General Elective (GE) Courses}

\begin{tabularx}{\textwidth}{|X|l|}
\hline 
\textbf{Credits Offered} & 17 \\ \hline 
\textbf{Credits Earned from General Elective courses to claim the Degree} & 14 \\ \hline 
\textbf{Credit Deficit from General Elective courses} & 3 \\
\hline 
\end{tabularx}

\vspace{10pt}

\noindent The General Elective courses do not count towards GPA calculation. But these courses count for the Earned Credits to claim the degree and Credit Deficit calculations. If a course is repeated, the grade obtained in the best attempt is used for all of the above calculations.

\begin{table}[H]
\begin{tabularx}{\textwidth}{
    |>{\hsize=0.8\hsize}X| 
    >{\hsize=0.6\hsize}X|
    >{\hsize=2.6\hsize}X|
    >{\hsize=0.5\hsize}X|
    >{\hsize=0.5\hsize}X|
  }
\hline
\textbf{Semester Ending Date} & \textbf{Course ID} & \textbf{Course Unit Name} & \textbf{Grade} & \textbf{Credits} \\ 
\hline
29-Dec-17 & EF501 & The Engineer in Society & A+ & 2 \\ 
\hline
29-Dec-17 & EF509 & Engineer as an Entrepreneur & A & 3 \\ 
\hline
29-Dec-17 & EF524 & Business Law & A+ & 3 \\ 
\hline
29-Dec-17 & EF528 & Introduction to Digital Art & A+ & 3 \\ 
\hline


\end{tabularx}
\end{table}

\section[GP102: English II and TR400: Industrial Training Courses]{\texorpdfstring{GP102: English II and TR400: Industrial Training \\Courses}{GP102: English II and TR400: Industrial Training Courses}}

TR400: Industrial Training will have to be Passed for the student to successfully complete the Specialization Programme in Engineering. After passing TR400, the student earns 6 more credits towards claiming the degree.

\noindent GP102: English II will have to be Passed for the student to successfully complete the General Programme in Engineering. After passing GP102, the student earns 3 more credits towards claiming the degree.

\begin{table}[H]
\begin{tabularx}{\textwidth}{|L|l|l|l|l|}
\hline 
\textbf{Semester Ending Date} & \textbf{Course ID} & \textbf{Course Unit Name} & \textbf{Grade} & \textbf{Credits} \\ 
\hline
13-Oct-16 & GP102 & English II & PASS & 3 \\ 
\hline
\end{tabularx}
\end{table}


%\noindent
%\begin{minipage}[t]{0.4\linewidth}
%\centering
%
%\begin{table}[H]
%\begin{tabularx}{1.0\textwidth}{|L|L|}
%\hline
%\textbf{Grade} & \textbf{Points} \\ 
%\hline
%A+ & 4.0 \\ 
%A & 4.0 \\ 
%A- & 3.7 \\ 
%B+ & 3.3 \\ 
%B & 3.0 \\ 
%B- & 2.7 \\ 
%C+ & 2.3 \\ 
%\hline
%\end{tabularx}
%\end{table}
%
%\end{minipage}%
%\hfill
%  \begin{minipage}[t]{0.4\linewidth}
%  
%\begin{table}[H]
%\begin{tabularx}{1.0\textwidth}{|L|L|}
%\hline
%\textbf{Grade} & \textbf{Points} \\ 
%\hline
%C & 2.0 \\ 
%C- & 1.7 \\ 
%D+ & 1.3 \\ 
%D & 1.0 \\ 
%E & 0.7 \\ 
%F & 0.0 \\ 
%\hline
%\end{tabularx}
%\end{table}  
%\end{minipage}


\noindent\rule{\textwidth}{1pt}
\vspace{5pt}
This is an interim academic transcript issued at the request of the student, covering seven (7) semesters that have been completed so far.
\vspace{5pt}

\noindent\textbf{Effective date:} 15-Feb-19

\vspace{45pt}

\noindent
\begin{minipage}[t]{0.5\linewidth}
    \raggedright
    \sign{Prof. Roshan G. Ragel}
    \par
    Head of Department\par
    Department of Computer Engineering, \par
    Faculty of Engineering, \par
    University of Peradeniya, Sri Lanka
\end{minipage}%
\hfill
  \begin{minipage}[t]{0.4\linewidth}
    \Date
  \end{minipage}



\textbf{Note:} Grade Points are given according to 0.0 - 4.0 scale

\begin{table}[H]
\centering
\begin{tabular}{ll}
\multicolumn{1}{c}{Grade} &
\multicolumn{1}{c}{Points}\\
% Grade & Points \\ 
\ \ A+    & \ \ 4.0    \\ 
\ \ A     & \ \ 4.0    \\ 
\ \ A-    & \ \ 3.7    \\ 
\ \ B+    & \ \ 3.3    \\ 
\ \ B    & \ \ 3.0    \\ 
\ \ B-    & \ \ 2.7    \\ 
\ \ C+    & \ \ 2.3    \\ 
\ \ C     & \ \ 2.0    \\ 
\ \ C-    & \ \ 1.7    \\ 
\ \ D+    & \ \ 1.3    \\ 
\ \ D     & \ \ 1.0    \\ 
\ \ E     & \ \ 0.0    \\ 
\end{tabular}
\end{table}

\begin{center}
**End of the document**
\end{center}


\end{document}
